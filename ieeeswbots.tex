% This is "sig-alternate.tex" V2.0 May 2012
% This file should be compiled with V2.5 of "sig-alternate.cls" May 2012
%
% This example file demonstrates the use of the 'sig-alternate.cls'
% V2.5 LaTeX2e document class file. It is for those submitting
% articles to ACM Conference Proceedings WHO DO NOT WISH TO
% STRICTLY ADHERE TO THE SIGS (PUBS-BOARD-ENDORSED) STYLE.
% The 'sig-alternate.cls' file will produce a similar-looking,
% albeit, 'tighter' paper resulting in, invariably, fewer pages.
%
% ----------------------------------------------------------------------------------------------------------------
% This .tex file (and associated .cls V2.5) produces:
%       1) The Permission Statement
%       2) The Conference (location) Info information
%       3) The Copyright Line with ACM data
%       4) NO page numbers
%
% as against the acm_proc_article-sp.cls file which
% DOES NOT produce 1) thru' 3) above.
%
% Using 'sig-alternate.cls' you have control, however, from within
% the source .tex file, over both the CopyrightYear
% (defaulted to 200X) and the ACM Copyright Data
% (defaulted to X-XXXXX-XX-X/XX/XX).
% e.g.
% \CopyrightYear{2007} will cause 2007 to appear in the copyright line.
% \crdata{0-12345-67-8/90/12} will cause 0-12345-67-8/90/12 to appear in the copyright line.
%
% ---------------------------------------------------------------------------------------------------------------
% This .tex source is an example which *does* use
% the .bib file (from which the .bbl file % is produced).
% REMEMBER HOWEVER: After having produced the .bbl file,
% and prior to final submission, you *NEED* to 'insert'
% your .bbl file into your source .tex file so as to provide
% ONE 'self-contained' source file.
%
% ================= IF YOU HAVE QUESTIONS =======================
% Questions regarding the SIGS styles, SIGS policies and
% procedures, Conferences etc. should be sent to
% Adrienne Griscti (griscti@acm.org)
%
% Technical questions _only_ to
% Gerald Murray (murray@hq.acm.org)
% ===============================================================
%
% For tracking purposes - this is V2.0 - May 2012

\documentclass{sig-alternate}
\usepackage[normalem]{ulem}
\usepackage{graphics}
\usepackage{graphicx}
\usepackage{hyperref}

% playing with boxes! and colors to improve presentation of practices.
\usepackage{framed,color}
\usepackage{xcolor}
\usepackage{csquotes}
\usepackage{array}

\usepackage{footmisc}

%change this to 
\usepackage[turnoff]{notes}
% to make all of the notes disappear
%\usepackage{notes}

% create macros for each person based on initials so they can easily
% add comments to the paper.
\newcommand{\cb}[1]{\note[Carly]{#1}}
\newcommand{\ms}[1]{\note[Peggy]{\todo{(Peggy) #1}}}
\newcommand{\lm}[1]{\note[Alexey]{#1}}
\newcommand{\mg}[1]{\note[Matthieu]{#1}}
\newcommand{\jc}[1]{\note[Cassie]{\todo{(Cassie) #1}}}

\useunder{\uline}{\ul}{}
\begin{document}
%
% --- Author Metadata here ---
\conferenceinfo{WOODSTOCK}{'97 El Paso, Texas USA}
%\CopyrightYear{2007} % Allows default copyright year (20XX) to be over-ridden - IF NEED BE.
%\crdata{0-12345-67-8/90/01}  % Allows default copyright data (0-89791-88-6/97/05) to be over-ridden - IF NEED BE.
% --- End of Author Metadata ---

%\title{Should ChatBots play a role in your software development project?}
%\title{Two Developer Perspectives on Bots:  Creation and Use}
% or
% \title{How Bots may Play a Role in Your Development Project}
\title{A New Generation of Software Bots:  \\
How Developers can Create and Use Them}




%Format\titlenote{(Produces the permission block, and
%copyright information). For use with
%SIG-ALTERNATE.CLS. Supported by ACM.}}
% \subtitle{[Extended Abstract]
%\titlenote{A full version of this paper is available as
%\textit{Author's Guide to Preparing ACM SIG Proceedings Using
%\LaTeX$2_\epsilon$\ and BibTeX} at
%\texttt{www.acm.org/eaddress.htm}}}
%
% You need the command \numberofauthors to handle the 'placement
% and alignment' of the authors beneath the title.
%
% For aesthetic reasons, we recommend 'three authors at a time'
% i.e. three 'name/affiliation blocks' be placed beneath the title.
%
% NOTE: You are NOT restricted in how many 'rows' of
% "name/affiliations" may appear. We just ask that you restrict
% the number of 'columns' to three.
%
% Because of the available 'opening page real-estate'
% we ask you to refrain from putting more than six authors
% (two rows with three columns) beneath the article title.
% More than six makes the first-page appear very cluttered indeed.
%
% Use the \alignauthor commands to handle the names
% and affiliations for an 'aesthetic maximum' of six authors.
% Add names, affiliations, addresses for
% the seventh etc. author(s) as the argument for the
% \additionalauthors command.
% These 'additional authors' will be output/set for you
% without further effort on your part as the last section in
% the body of your article BEFORE References or any Appendices.

\numberofauthors{1} %  in this sample file, there are a *total*
% of EIGHT authors. SIX appear on the 'first-page' (for formatting
% reasons) and the remaining two appear in the \additionalauthors section.
%
\author{
% You can go ahead and credit any number of authors here,
% e.g. one 'row of three' or two rows (consisting of one row of three
% and a second row of one, two or three).
% 
% The command \alignauthor (no curly braces needed) should
% precede each author name, affiliation/snail-mail address and
% e-mail address. Additionally, tag each line of
% affiliation/address with \affaddr, and tag the
% e-mail address with \email.
%
% 1st. author
\alignauthor
XXX\\
       \affaddr{XXX}\\
 % }    
}
% There's nothing stopping you putting the seventh, eighth, etc.
% author on the opening page (as the 'third row') but we ask,
% for aesthetic reasons that you place these 'additional authors'
% in the \additional authors block, viz.
%\additionalauthors{Additional authors: John Smith (The Th{\o}rv{\"a}ld Group,
%email: {\texttt{jsmith@affiliation.org}}) and Julius P.~Kumquat
%(The Kumquat Consortium, email: {\texttt{jpkumquat@consortium.net}}).}

\date{29 Sept 2017}
% Just remember to make sure that the TOTAL number of authors
% is the number that will appear on the first page PLUS the
% number that will appear in the \additionalauthors section.

\maketitle
% Abstract
% no abstract for a magazine article is needed
%\input{Abstract.tex}

% A category with the (minimum) three required fields
% \category{A.B}{Category}{Miscellaneous}
%A category including the fourth, optional field follows...
% \category{A.B.C}{Some Category}{Sub category}[term 1, term 2]

% \terms{Term 1, Term 2, Term 3}

% \keywords{ACM proceedings, \LaTeX, text tagging}

\section{The Emergence of Bots} 

% [Peggy]

% outline: the idea of turing test...
From the earliest days of computer programs, developers have imagined the emergence of programs that can act, talk and think like humans.
Such programs would not only automate tasks that humans might do, but they could also work with humans to solve intellectual tasks that cannot be entirely automated. 
The hope, even as far back as 1966, was for these programs to pass the Turing Test (proposed in the 1950 paper Computing Machinery and Intelligence [s]), where humans could be fooled into believing that they are interacting with an intelligent human rather than a mere program (reference Eliza Bot [x]). 

% outline: what is a chatbot or bot, and not about turing these days
The terms ``chatbot'' and ``bot'' were interchangeably used to describe the realization of this vision quite early on, but today they are used to refer to a conversational style user interface or an anthropomorphized script or agent that can automate rote and tedious tasks.
Bots are not (typically) intended to fool the end user into believing a bot is a real person, but many bots do have a personality that is engaging and pleasant to interact with. 
 
% outline:  where they reside and briefly what they do
Bots typically reside on popular platforms where users work or play with other users, and frequently integrate other services and micro-services into these channels, providing a conduit between users and other software services. 
They may fetch or share information, extract and analyze data, detect and monitor events and activities in communication and social media, connect users with each other or with other tools, or they may provide feedback and recommendations on individual and collaborative tasks. 

% outline: its time is now ... messaging platforms and also NLP
Bots are rapidly becoming a \emph{defacto} interface for interacting with software services in just the past few years.  In part, this is because of the widespread adoption of messaging platforms (e.g., Facebook for social networking and Slack for developer communication) and in part because of the advancement of natural language processing, which many bots (but not all) leverage.
% outline: bring in AI and data
But another driver is the prevalence of ``big data'' along with machine learning algorithms for analyzing data across many domains.  Bots provide a convenient way for developers to generate a user interface for interacting with these algorithms and data. 

% outline: intuitive progression
Developers use bots as well as create them. For developers, the transition from command line interfaces to interacting with bots through the messaging interface feels intuitive and has the advantage of bringing transparency of invoking and customizing services in a communication channel, while non-technical users are also embracing the notion of bots as opposed to installing and relying on apps that are not well integrated with their messaging environment.  

% what other companies are doing
All of the major software software companies clearly recognize the value that bots bring in terms of integration of services, users and communication channels. Facebook aims to ``replace apps'' one bot at a time in their messaging platform [x], while Microsoft claims that ``conversation as a platform'' is the operating system of the future rather than Windows [x]. 
Amazon's Alexa, Apple's Siri and Google's Now platform are also showing agreement with this rapid shift towards bots.
Software developers will also recognize many bots in the platforms they frequently use to connect with other developers and services, such as Slack [x], Teams [x] and HipChat [x]. 

% outline and main point in the rest of the column
In this column, we first describe the key characterstics of bots and how they can be used to automate many end user tasks across a wide variety of user domains.  
We next provide background for developers on how they can create and host bots on various platforms. These platforms support one or more software frameworks that developers can leverage to support the rapid creation and design of bots. We also discuss some design guidelines for bots, pulling ideas from the emerging area of human-robot interaction. 
In a sidebar, we look at how developers themselves use bots in their development work.  We call out this domain because to quote XXX from StackOverflow, ``developers are writing the script for tomorrow'' and thus we see examples of sophisticated and innovative bots emerging from developer needs that pave the way for how bots may be created for other domains.
Finally, we conclude this column with some considerations and advice for developers wishing to develop bots for their end users or wishing to use bots within their own development projects.   

  

\section{Botology: Understanding the landscape of bots}

% [Peggy]

\todo{improve and add more examples}


It has been surprising to see the rapid development and widespread adoption of bots in just a few years, but what is more surprising is the vast array of bots that support so many diverse tasks and roles.  
Rather than attempting to narrowly define bots or chatbots, instead we look at how they may be characterized. 

One way to characterize bots is through the \textbf{interaction model} they provide. 
Some bots support a domain specific language where the user interacts with the bot using specific commands within a command line interface.  For example, the Felix bot\footnote{\url{http://felixbot.co/}} (which integrates in Slack) boosts personal productivity and a user interacts with Felix using commands such as ``start'' and ``done'' to tell Felix about the status of their personal tasks.   
Other bots rely tend to rely on natural language processing, such as Apple's Siri.
Bots also differ in terms of whether they support a pull based approach, where the user initiates the interaction (e.g., a user invokes the bot using commands using ``Hey Siri''), or the bot may initiate the interaction based on some system or user context (e.g., the Poncho weather bot\footnote{\url{https://poncho.is/}} which can be used across several platforms including Facebook Messenger, Viber, Android and iPhone, will warn the user when it is about to rain). 
% does Poncho do that? I think it does...

Another way to characterize bots is in terms of their \textbf{intelligence} which we further can characterize as follows:
\todo{Carly to expand on these bullets}
\begin{itemize}
\item \emph{Adaptation:} How ``context aware'' the bot is and how the bot uses that context to change how it interacts with the user(s).  For example, ....
\item \emph{Reasoning:} Some bots follow very simple logic rules (e.g., IFTTT) whereas others bots use more advanced Artificial Intelligence (AI).  For example, ...
\item \emph{Autonomy:}  Some bots may be entirely autonomous whereas other bots rely on a human user's input.  For example, ....
\end{itemize}



Finally, bots may be characterized according to their \textbf{purpose}, as we summarize here: 
\begin{itemize}
\item \emph{transactional} bots provide support for user transactions such as...
\item \emph{informational} bots exist to provide critical information to its users.  For example, the SwingTradeBot\footnote{\url{https://swingtradebot.com/}} helps its users stay on top of the stock market, by watching stocks and scanning the market for important developments and then alerting its users when they may need to take an action on their portfolio. 
\item \emph{productivity} bots aim to improve a user's or team's productivity by automating rote or tedious tasks such as updating a calendar automatically when you email the bot (see \url{x.ai}), or more sophisticated bots such as Tomatobot\footnote{\url{https://tomatobot.matthewhiggins.me/}} that integrates in Slack that help you follow the Pomodoro Method\footnote{\url{https://blog.trello.com/how-to-pomodoro-your-way-to-productivity}} for boosting one's productivity. 
\item \emph{collaboration} bots support how users communicate, coordinate or collaborate their tasks.  For example, the Meekan bot\footnote{\url{https://meekan.com/}} helps users coordinate and arrange their schedules to facilitate meetings.  Meekan integrates across a number of different platforms including Slack, Hipchat and Microsoft Teams.
% I have too many software development examples? 
\end{itemize}
	
	
\section{How to create Bots and where to host them}

	[Carly]

	Although simple bots can be built from scratch and self-hosted, many developers choose to leverage third-party technologies to streamline the process. With the explosion of new tools to match in the bot development domain, if is important to distinguish between the tools used to build bots (Creation Platforms) and the platforms where they dwell (Distribution Platforms).

	Many large companies, such as Microsoft and Facebook for example, offer a comprehensive set of tools to support both the creation and distribution of bots. Other companies, provide specialized solutions for specific aspects of bot creation and distribution. 

	\subsection{Distribution Platforms}
	The distribution platforms define where and how the bots are accessed by users. They range from general purpose social networks (e.g. Facebook Messenger) to domain specific channels (e.g. Slack, Teams, HipChat).  

	Distribution platforms offer many benefits for developers: access to an existing user base\footnote{https://medium.com/mobile-lifestyle/messenger-vs-skype-vs-slack-vs-telegram-how-to-spot-the-winners-adc34b4ca066\label{How_to_spot_the_winners}}, defined interaction mechanisms, discovery\footref{How_to_spot_the_winners}, and monetization\footref{How_to_spot_the_winners}.  

	Launching a bot on an existing platform, helps overcome the cold start problem that many new applications face. Selecting the wrong user base, however, can be equally detrimental. Carefully consider the size and demographics of the user base as well as the cost to access the platform. 

	Distribution platforms also define and standardize how users can interaction with bots.  Commonly support interaction mechanisms include: text, user can communicate with the bots using natural language; commands, users can trigger actions with a set of keywords or phrases; selection, users can choose from a set of presented options; and voice, users can talk to the bots.

	Distribution platforms also offer mechanisms promotion or for user discovery of the bots. Similar to Apple's App-Store [x], many distribution platforms offer virtual ``bot stores'' where users browse for bots. Third-party sites (e.g. BotList) also offer online catalogs of bots for many popular distribution platforms. 

	Mature distribution channels may also offer means of collecting payment from users, particularly usefully when developing transactional bots.

	A summary of popular distribution platforms and technologies is included in figure X.

	\subsection{Creation Platforms}
	Developer ecosystem for the different platforms...  plug in to these communities...

	The creation platforms support the design and development of bots. This support often comes in the form of software foundations (e.g. frameworks, tool kits, APIs, or other services). These bot creation services can be platform specific or produce bots that can be deployed across multiple platforms (Microsoft Bot Connector). The services they provide range from documentation and code templates, to no-code-required bot building interfaces. 
	Many of the popular bot creation platforms also offer vibrant developer ecosystems. Developers can plug into these communities for developer expertise in the form of tutorials, articles, discussions, and support.

	A summary of popular creation platforms and technologies is included in figure X.

	\subsection{Design guidelines}


\section{Takeaways}

[Peggy]
These days, bots are becoming pervasive.  End users experience them in cars, in the home, in entertainment devices and at work.  We see them particularly playing a significant and sophisticated role in many software development projects (as we discussed in the side bar).  
We suggest that developers learn from these experiences and watch for innovative ways to deploy them in their own projects (to improve their own activities but also as a feature for their end users), but at the same time, not to overuse them! 
We suggest that careful attention should be paid to the challenges they may introduce including information overload and interruptions.  
	
	[Carly]
Another challenge they may introduce includes privacy intrusion. 
	Bot ethics:  Reference Asimov's rules for Robots here..   Should this be updated for Bots?
	
	[Carly]
	Bots may ease or increase collaboration friction...  choose Bots not to remove collaboration (and in turn creativity) but use them to enhance collaboration (reference CSCW workshop paper)



\section{Sidebar}

[Peggy]

Software bot usage in software development.
(reference FSE visions paper)




%ACKNOWLEDGMENTS are optional
% Acknowledgements
%\input{Acknowledgements.tex}



%
% The following two commands are all you need in the
% initial runs of your .tex file to
% produce the bibliography for the citations in your paper.
\bibliographystyle{IEEEtran}
\bibliography{IEEEabrv,ieeesw2016} % sigproc.bib is the name of the Bibliography in this case

%\input{bios.tex}

\listoftodos
% You must have a proper ".bib" file
%  and remember to run:
% latex bibtex latex latex
% to resolve all references
%
% ACM needs 'a single self-contained file'!
%
%\balancecolumns % GM June 2007
% That's all folks!
\end{document}
