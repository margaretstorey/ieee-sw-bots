% This is "sig-alternate.tex" V2.0 May 2012
% This file should be compiled with V2.5 of "sig-alternate.cls" May 2012
%
% This example file demonstrates the use of the 'sig-alternate.cls'
% V2.5 LaTeX2e document class file. It is for those submitting
% articles to ACM Conference Proceedings WHO DO NOT WISH TO
% STRICTLY ADHERE TO THE SIGS (PUBS-BOARD-ENDORSED) STYLE.
% The 'sig-alternate.cls' file will produce a similar-looking,
% albeit, 'tighter' paper resulting in, invariably, fewer pages.
%
% ----------------------------------------------------------------------------------------------------------------
% This .tex file (and associated .cls V2.5) produces:
%       1) The Permission Statement
%       2) The Conference (location) Info information
%       3) The Copyright Line with ACM data
%       4) NO page numbers
%
% as against the acm_proc_article-sp.cls file which
% DOES NOT produce 1) thru' 3) above.
%
% Using 'sig-alternate.cls' you have control, however, from within
% the source .tex file, over both the CopyrightYear
% (defaulted to 200X) and the ACM Copyright Data
% (defaulted to X-XXXXX-XX-X/XX/XX).
% e.g.
% \CopyrightYear{2007} will cause 2007 to appear in the copyright line.
% \crdata{0-12345-67-8/90/12} will cause 0-12345-67-8/90/12 to appear in the copyright line.
%
% ---------------------------------------------------------------------------------------------------------------
% This .tex source is an example which *does* use
% the .bib file (from which the .bbl file % is produced).
% REMEMBER HOWEVER: After having produced the .bbl file,
% and prior to final submission, you *NEED* to 'insert'
% your .bbl file into your source .tex file so as to provide
% ONE 'self-contained' source file.
%
% ================= IF YOU HAVE QUESTIONS =======================
% Questions regarding the SIGS styles, SIGS policies and
% procedures, Conferences etc. should be sent to
% Adrienne Griscti (griscti@acm.org)
%
% Technical questions _only_ to
% Gerald Murray (murray@hq.acm.org)
% ===============================================================
%
% For tracking purposes - this is V2.0 - May 2012

\documentclass{sig-alternate}
\usepackage[normalem]{ulem}
\usepackage{graphics}
\usepackage{graphicx}
\usepackage{hyperref}

% playing with boxes! and colors to improve presentation of practices.
\usepackage{framed,color}
\usepackage{xcolor}
\usepackage{csquotes}
\usepackage{array}

\usepackage{footmisc}

%change this to 
\usepackage[turnoff]{notes}
% to make all of the notes disappear
%\usepackage{notes}

% create macros for each person based on initials so they can easily
% add comments to the paper.
\newcommand{\cb}[1]{\note[Carly]{#1}}
\newcommand{\ms}[1]{\note[Peggy]{\todo{(Peggy) #1}}}
\newcommand{\lm}[1]{\note[Alexey]{#1}}
\newcommand{\mg}[1]{\note[Matthieu]{#1}}
\newcommand{\jc}[1]{\note[Cassie]{\todo{(Cassie) #1}}}

\useunder{\uline}{\ul}{}
\begin{document}
%
% --- Author Metadata here ---
\conferenceinfo{WOODSTOCK}{'97 El Paso, Texas USA}
%\CopyrightYear{2007} % Allows default copyright year (20XX) to be over-ridden - IF NEED BE.
%\crdata{0-12345-67-8/90/01}  % Allows default copyright data (0-89791-88-6/97/05) to be over-ridden - IF NEED BE.
% --- End of Author Metadata ---

\title{Should ChatBots play a role in your software development project}




%Format\titlenote{(Produces the permission block, and
%copyright information). For use with
%SIG-ALTERNATE.CLS. Supported by ACM.}}
% \subtitle{[Extended Abstract]
%\titlenote{A full version of this paper is available as
%\textit{Author's Guide to Preparing ACM SIG Proceedings Using
%\LaTeX$2_\epsilon$\ and BibTeX} at
%\texttt{www.acm.org/eaddress.htm}}}
%
% You need the command \numberofauthors to handle the 'placement
% and alignment' of the authors beneath the title.
%
% For aesthetic reasons, we recommend 'three authors at a time'
% i.e. three 'name/affiliation blocks' be placed beneath the title.
%
% NOTE: You are NOT restricted in how many 'rows' of
% "name/affiliations" may appear. We just ask that you restrict
% the number of 'columns' to three.
%
% Because of the available 'opening page real-estate'
% we ask you to refrain from putting more than six authors
% (two rows with three columns) beneath the article title.
% More than six makes the first-page appear very cluttered indeed.
%
% Use the \alignauthor commands to handle the names
% and affiliations for an 'aesthetic maximum' of six authors.
% Add names, affiliations, addresses for
% the seventh etc. author(s) as the argument for the
% \additionalauthors command.
% These 'additional authors' will be output/set for you
% without further effort on your part as the last section in
% the body of your article BEFORE References or any Appendices.

\numberofauthors{1} %  in this sample file, there are a *total*
% of EIGHT authors. SIX appear on the 'first-page' (for formatting
% reasons) and the remaining two appear in the \additionalauthors section.
%
\author{
% You can go ahead and credit any number of authors here,
% e.g. one 'row of three' or two rows (consisting of one row of three
% and a second row of one, two or three).
% 
% The command \alignauthor (no curly braces needed) should
% precede each author name, affiliation/snail-mail address and
% e-mail address. Additionally, tag each line of
% affiliation/address with \affaddr, and tag the
% e-mail address with \email.
%
% 1st. author
\alignauthor
XXX\\
       \affaddr{XXX}\\
 % }    
}
% There's nothing stopping you putting the seventh, eighth, etc.
% author on the opening page (as the 'third row') but we ask,
% for aesthetic reasons that you place these 'additional authors'
% in the \additional authors block, viz.
%\additionalauthors{Additional authors: John Smith (The Th{\o}rv{\"a}ld Group,
%email: {\texttt{jsmith@affiliation.org}}) and Julius P.~Kumquat
%(The Kumquat Consortium, email: {\texttt{jpkumquat@consortium.net}}).}

\date{28 Sept 2017}
% Just remember to make sure that the TOTAL number of authors
% is the number that will appear on the first page PLUS the
% number that will appear in the \additionalauthors section.

\maketitle
% Abstract
% no abstract for a magazine article is needed
%\input{Abstract.tex}

% A category with the (minimum) three required fields
% \category{A.B}{Category}{Miscellaneous}
%A category including the fourth, optional field follows...
% \category{A.B.C}{Some Category}{Sub category}[term 1, term 2]

% \terms{Term 1, Term 2, Term 3}

% \keywords{ACM proceedings, \LaTeX, text tagging}

\section{What are Bots and their History} 
	[Peggy]

	(shift from ``human'' imposter to being recognized as ``scripts'' or ``agents'' that are conversational in style)  

\section{Why are they so popular now}
	
	(timing of platforms and chat environments)
	
	Is this really new?  

\section{Botology: Types / Roles of bots}
	Not just development roles, more general roles (e.g., transactional, informational, productivity, collaboration...)
	
	Give general examples as well as some specific SE examples throughout 


\section{How to create Bots and where to host them}
	[Carly]

	There is a need to distinguish between the services used to build bots (Creation Platforms) and the platforms where they live (Distribution Platforms).  Although simple chatbots can be built from scratch and self-hosted, many developers choose to leverage a variety of third-party technologies to streamline the process.

	Many large companies, such as Microsoft and Facebook for example, offer a comprehensive set of tools to support both the creation and distribution of bots. Other companies, provide specialized solutions for specific aspects of bot creation and distribution. 

	\subsection{Distribution Platforms}
	The distribution platforms are where the bots can be accessed by users. Bot distribution platforms can be general purpose, such as FB Messenger, or specific to a domain such as Slack in Software Development.  

	Distribution platforms offer many benefits to developers: access to a user base\footnote{https://medium.com/mobile-lifestyle/messenger-vs-skype-vs-slack-vs-telegram-how-to-spot-the-winners-adc34b4ca066\label{How_to_spot_the_winners}}, defined interactions, marketing\footref{How_to_spot_the_winners}, and collecting payments\footref{How_to_spot_the_winners}.  Launching your bot on a platform that has a stable user base helps overcome the cold start problem that many new applications face. When selecting a user base, consider the size / demographics of the user base and user cost to access platform. The distribution platforms also define and standardize how users can interaction with the bots.  The most common bot interaction mechanisms include text (responding to the bot in natural language), commands (using a set of predefined keywords or phrases), selection (directly clicking on one of the presented options), and sometimes voice.

	Distribution platforms also offer the ability to market or promote the bots. Popular distribution platforms offer ``bot stores'' where users can go and shop for bots. Many third-party also sites that maintain catalogs of bots across many distribution platforms.  

	Lastly, many distribution channels offer means of collecting payments from users.  This is particularly usefully for transactional bots.

	A summary of the popular distribution platforms and technologies are included in figure X.

	\subsection{Creation Platforms}
	Developer ecosystem for the different platforms...  plug in to these communities...

	The creation platforms are where the bots are designed and built. Creation platforms offer many benefits to developers:  support for specific distribution channels, and a access to an active development community.  A successful creation platform offers software foundations in the form of APIs, tool sets, services, or frameworks and a developer ecosystem for support\footref{How_to_spot_the_winners}. The creation platforms can support the development of bots for specific distribution platforms, or many produce general bots that can be deployed on multiple platforms.  The services they provide range from documentation and code templates to no-code required bot building interfaces.  Surrounding the creation platforms are developers using the same technologies, driving innovation and new use cases for bots.  They provide software expertise in the form of tutorials, articles, discussions, and answering questions.

	A summary of the popular creation platforms and technologies are included in figure X.

	\subsection{Design guidelines}


\section{Takeaways}
	Bots are everywhere: pay attention to how bots are used around you, in cars, in the home but also watch how other developers use Bots in their development activities (reference FSE visions paper)
	
	Bot ethics:  Reference Asimov's rules for Robots here..   Should this be updated for Bots?
	
	Bots may ease or increase collaboration friction...  choose Bots not to remove collaboration (and in turn creativity) but use them to enhance collaboration (reference CSCW workshop paper)



\section{Sidebar}

Software bot usage in software development.




%ACKNOWLEDGMENTS are optional
% Acknowledgements
%\input{Acknowledgements.tex}



%
% The following two commands are all you need in the
% initial runs of your .tex file to
% produce the bibliography for the citations in your paper.
\bibliographystyle{IEEEtran}
\bibliography{IEEEabrv,ieeesw2016} % sigproc.bib is the name of the Bibliography in this case

%\input{bios.tex}

\listoftodos
% You must have a proper ".bib" file
%  and remember to run:
% latex bibtex latex latex
% to resolve all references
%
% ACM needs 'a single self-contained file'!
%
%\balancecolumns % GM June 2007
% That's all folks!
\end{document}
